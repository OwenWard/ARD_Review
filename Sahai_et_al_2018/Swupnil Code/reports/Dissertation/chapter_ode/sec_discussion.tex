\section{Discussion}
\label{sec:ode_discussion}

This paper presents an improved statistical method of learning cosmological parameters from type-Ia supernovae. We show that the Stan programming language, combined with an improved set of priors, provides massive computational gains in both speed and stability over current approaches.

Recent work has shown that cosmological parameters can be learned by using a clever hierarchical model that pools various supernovae-related covariates together. In particular, \citet{Shariff+others:2016} show that a Gibbs sampler can be applied to a series of models containing various interesting covariates, resulting in more precise estimates of the cosmological parameters. This implementation, however, uses rigid uniform priors, and also takes multiple days to converge to the posterior. In contrast, the NUTS implementation in Stan uses more flexible normally distributed priors and a fourth and fifth order Runge-Kutta method in C++ to converge to the posterior in just a few hours. Perhaps most crucially, the Stan code for the model is easily interpretable because of its probabilistic nature, and its results can be easily reproduced with the model file spanning less than 250 lines. 

Despite the improvements presented here, numerous other changes could be made to the BAHAMAS model. The heliocentric and cosmic microwave background adjusted redshifts, for example, could be incorporated into a measurement error model instead of simply treating the observed values as the true redshifts. While this would normally cause numerical issues when computing the integrated luminosity (because redshift is the upper limit of the integral), our change of variables allows the integral's limits to be constants, making the computation more tractable. Additionally, the empirical covariance matrix of the peak magnitude, stretch correction, and color correction could be incorporated into a measurement error model as well. In particular, methods using Wishart (not inverse-Wishart) priors have shown guarantees of positive definite posterior modes \citep{Chung+others:2015}.