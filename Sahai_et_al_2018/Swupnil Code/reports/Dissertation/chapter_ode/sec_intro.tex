\section{Introduction}
\label{sec:ode_introduction}

Supernovae, the astronomical events at the end of a massive star's life marked by a catastrophic explosion, have tremendous potential in helping researchers learn instrinsic properties about the entire universe. In particular, supernovae of type Ia (SNe Ia) have been instrumental in establishing the accelerated expansion of the universe, starting with the unexpected discovery by the Supernova Cosmology Project \citep{Riess+others:1998} and the High-z Supernova Search Team \citep{Perlmutter+others:1999}. Because SNe Ia emit radiation that probes the low-redshift (i.e. less visible) universe, they are ideal tools to measure the properties of "dark energy", a form of energy that permeates all of space and tends to accelerate the expansion of the universe. In the last decade, the sample of SNe Ia has increased dramatically (e.g., \citet{Wood-Vasey+others:2007}; \citet{Kowalski+others:2008}; \citet{Kessler+others:2009}; \citet{Contreras+others:2010}; \citet{Suzuki+others:2012}; \citet{Betoule+others:2014}), and it now comprises several hundred spectroscopically confirmed SNe Ia. 

SNe Ia occur when material from a companion star accreting onto a white dwarf star triggers carbon fusion, proceeding until a core of typical mass $0.7\solarmass$ of $^{56}$Ni is created (the mass of our sun is $1\solarmass = 1.99 \times 10^{30}$kg). After a type Ia supernova's explosion, its luminosity can be observed over time, creating a  light curve (LC). Within the more restricted subclass of so-called "normal" SNe Ia, the fundamental assumption underlying their use to measure expansion history is that they can be standardized so that their peak luminosity magnitude (in the LC) are sufficiently homogeneous. The relative uniformity of these intrinsic magnitudes allows us to measure the distance of the SNe Ia from their host galaxy (known as distance modulus) because a type Ia supernova's observed peak magnitude depends primarily on its intrinsic magnitude and its distance modulus.

A supernova's distance modulus is of utmost importance to estimate because it is a function of integrated luminosity, an integral whose value is determined by both the supernova's specific redshift and a global set of cosmological parameters (one of which is the dark energy component). One of the most widely used frameworks for estimating the distance modulus from LC data is the SALT2 method \citep{Guy+others:2005}, which derives color and stretch corrections for the magnitude from the LC fit, and then uses the corrected distance modulus to fit the underlying cosmological parameters. 

Recent work has applied the SALT2 methodology to a larger catalog of SNe Ia. \citet{March+others:2011} demonstrated with simulated data that a Bayesian hierarchical model has a reduced posterior uncertainty, smaller mean squared error, and better coverage properties than the standard approach. \citet{Betoule+others:2014} then reanalyzed 740 spectroscopically confirmed SNe Ia obtained by the SDSS-II and SNLS collaboration, known as the joint light curve analysis (JLA). Most recently, \citet{Shariff+others:2016} introduced BAHAMAS (BAyesian Hier-Archical Modeling for the Analysis of Supernova cosmology), an extension of the bayesian method first introduced by \citet{March+others:2011}, and applied it to the SNe Ia sample from the JLA. \citet{Shariff+others:2016} tested for evolution with redshift in the properties of SNe Ia, and investigated whether the posterior variance of the cosmological parameters can be reduced by exploiting correlations between the intrinsic magnitudes of SNe Ia and their host galaxy mass.

While the \citet{Shariff+others:2016} analysis is exhaustive in its investigation, there is room for statistical improvements. In this paper, we address some of the drawbacks of the \citet{Shariff+others:2016} implementation of BAHAMAS. In particular, we make adjustments to the priors, replacing rigid uniform and inverse gamma distributions and with normal distributions. We also demonstrate several computational improvements to the MCMC sampling by using the Stan probabilistic programming language \citep{Stan:2016}. Lastly, we investigate whether the residual scatter around the Hubble law can be further reduced by exploiting correlations between the intrinsic magnitudes of SNe Ia and their metallicity, formation rate, and host galaxy age.

This paper is organized as follows. In Section \ref{sec:ode_background} we introduce the SALT2 method as well as the original BAHAMAS methdology. In Section \ref{sec:ode_model} we describe some issues with the BAHAMAS framework, as well as our extensions and improvements. In Section \ref{sec:ode_results} we present results obtained when fitting our improved model in Stan, while conclusions appear in Section \ref{sec:ode_discussion}.