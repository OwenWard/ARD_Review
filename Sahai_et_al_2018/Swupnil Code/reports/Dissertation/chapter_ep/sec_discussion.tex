\section{Discussion}
\label{sec:ep_discussion}

This paper presents EP as a framework for distributed Bayesian inference of hierarchical models on partitioned data sets by using the principle of message passing with cavity and tilted distributions. We create an example EP algorithm that uses Stan \citep{Stan:2016} for the tilted distribution approximation, and we demonstrate on four data sets that distributed EP provides significant computational gains over the full MCMC approach.

In using EP for posterior inference, we assume that convergence will occur within a few iterations. For our three synthetic data sets and actual astronomy data set, this assumption is not violated. Convergence in general, however, is not guaranteed for EP. Additionally, we approach the inference problem in an exhaustive manner, trying various combinations of the number of sites $K$, number of data points per site $N_k$, and using a fixed precision smoothing value $\delta = 0.9$ throughout. In practice, it is not efficient to take such an exhaustive approach, as the computational cost of doing so far outweighs that of simply running full MCMC in the first place. Additional research is still required in order to determine efficient approaches to setting the optimal number of sites, smoothing value, etc.

Aside from finding a more efficient way to determine optimal tuning parameters for our particular implementation, we also believe the algorithm itself has room for improvement. In approximating the tilted distribution we assume that the tilted distribution is multivariate normal, which guides our estimation of its precision matrix. In the general case when the titled distribution is not normal, however, care must be taken to shrink the eigenvalues of the sample covariance matrix or impose sparse structure constraints on it \citep{Bodnar+others:2014,Friedman+others:2008}.

From the computational perspective, variations of the fully distributed EP approach are also worth exploring. For example, one could process the sites in parallel, but asynchronously, so that EP can move to the next iteration even faster. While this may decrease the amount of information available as a prior in the next iteration, the computational gains of the asynchronous approach may outweigh the costs in accuracy.
