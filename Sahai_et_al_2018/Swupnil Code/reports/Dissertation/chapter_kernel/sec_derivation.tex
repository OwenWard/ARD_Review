\section{Latent kernel representation of social mixing}
\label{sec:kernel_derivation}

As a way to place constraints on the elements of the mixing matrix, we attempt to model the social mixing rates using a more structured framework. In particular, we wish to replace the discrete histograms in Figure \ref{fig:nonrandom_mixing_matrix} with smooth continuous curves. In order to do this, we first treat age as continuous 
\begin{align}
a_i \in (-\infty,\infty)
\end{align}
rather than binning age into categories as in Equation \ref{eq:mix_age_cat}. Then, to create a standard functional form that can be consistent across all ego ages, we use a Gaussian kernel as our continuous curve. The result is that we replace the $42$ free parameters of the non-random mixing matrix with a non-random mixing kernel that both imposes structure and requires significantly fewer parameters to be estimated.

% Latent Mixing Kernel Model
\subsection{Latent mixing kernel}
\label{subsec:nonrandom_mixing_kernel}

Probabilistically, the biggest change we make is to use a Gaussian kernel to model the likelihood of alter $j$ being age $a_j$, given that ego $i$ with gender $g_i$ and age $a_i$ knows alter $j$ and $j$ has gender $g_j$. Hence, we have
\begin{align}
p(a_j | g_j, a_i, g_i, i \to j) 
&= \frac{ 1 }{ \sqrt{ 2\pi\lambda_{g_ig_j} } } \exp\biggl\{ -\frac{ (a_i - a_j)^2 }{ 2\lambda_{g_ig_j} } \biggr\} && \\\nonumber
&= \normal(a_j | a_i, \lambda_{g_ig_j}),
\end{align}
where $\lambda_{g_ig_j}$ is a latent variable that can be interpreted as the bandwidth of the age mixing kernel. Intuitively, small values of $\lambda_{g_ig_j}$ indicate that egos of gender $g_i$ only know alters of gender $g_j$ that are close to the ego in age, whereas larger values indicate the egos know alters of a wide range of ages. 

As for gender, if we assume that gender mixing does not depend on an ego's age, then we can use a gender-only mixing matrix to model the likelihood of alter $j$ being gender $g_j$, given that ego $i$ with gender $g_i$ and age $a_i$ knows alter $j$. Here we have
\begin{align}
p(g_j | a_i, g_i, i \to j) 
&= p(g_j | g_i, i \to j) && \\\nonumber
&= \rho_{g_ig_j},
\end{align}
where we constrain the rows of the gender mixing matrix to sum to 1 with
\begin{align}
\sum_{g_j} \rho_{g_ig_j} &= 1.
\end{align}

With the above quantities defined, the alter demographic distribution of an ego's network can now be modeled with rigid structure as
\begin{align}
p(a_j, g_j | a_i, g_i, i \to j)
&= p(g_j | a_i, g_i, i \to j) p(a_j | g_j, a_i, g_i, i \to j)  && \label{eq:nonrandom_mixing_kernel} \\\nonumber
&= \rho_{g_ig_j} \normal(a_j | a_i, \lambda_{g_ig_j}).
\end{align}

% Expectation Derivation
\subsection{Expectation derivation}
\label{subsec:nonrandom_mixing_kernel_expectation}

With the kernel framework outlined, we can derive the expected number of alters known in $\mathcal{G}_k$ by ego $i$ by replacing the sum over discrete $A_j$ in Derivation \ref{eq:nonrandom_mixing_matrix_derivation} with an integral over continuous $a_j$. The result is
\begin{align}
\mu_{ik} 
&= d_i \sum_{g_j} \int_{a_j} p( j \in \mathcal{G}_k, a_j, g_j | a_i, g_i, i \to j) da_j && \label{eq:nonrandom_mixing_kernel_derivation} \\\nonumber
&= d_i \sum_{g_j} \int_{a_j} p( j \in \mathcal{G}_k | a_j, g_j) p(a_j, g_j | a_i, g_i, i \to j) da_j && \\\nonumber
&= d_i \sum_{g_j} \int_{a_j} p( j \in \mathcal{G}_k | a_j, g_j) \rho_{g_ig_j} \normal(a_j | a_i, \lambda_{g_ig_j}) da_j && \\\nonumber
&= d_i \sum_{g_j} \rho_{g_ig_j} \int_{a_j} p( j \in \mathcal{G}_k | a_j, g_j) \normal(a_j | a_i, \lambda_{g_ig_j}) da_j && \\\nonumber
&= d_i \sum_{g_j} \rho_{g_ig_j} \int_{a_j} 
\biggl( \frac{ \int_{a} p( j \in \mathcal{G}_k | a, g_j) da }{ \int_{a} p( j \in \mathcal{G}_k | a, g_j) da } \biggr) 
p( j \in \mathcal{G}_k | a_j, g_j) \normal(a_j | a_i, \lambda_{g_ig_j}) da_j && \\\nonumber
&= d_i \sum_{g_j} \rho_{g_ig_j}
\biggl( \int_{a} p( j \in \mathcal{G}_k | a, g_j) da \biggr) 
\int_{a_j} \biggl( \frac{ p( j \in \mathcal{G}_k | a_j, g_j) }{ \int_{a} p( j \in \mathcal{G}_k | a, g_j) da } \biggr)
\normal(a_j | a_i, \lambda_{g_ig_j}) da_j && \\\nonumber
&\approx d_i \sum_{g_j} \rho_{g_ig_j} 
\biggl( \int_{a} p( j \in \mathcal{G}_k | a, g_j) da \biggr) 
\int_{a_j} \normal(a_j | \mu_{kg_j}, \sigma_{kg_j}^2)
\normal(a_j | a_i, \lambda_{g_ig_j}) da_j && \\\nonumber
&= d_i \sum_{g_j} \rho_{g_ig_j} 
\biggl( \int_{a} p( j \in \mathcal{G}_k | a, g_j) da \biggr) 
\normal(a_i | \mu_{kg_j}, \lambda_{g_ig_j} + \sigma_{kg_j}^2) && \\\nonumber
&= d_i \sum_{g_j} \rho_{g_ig_j} 
\biggl( \int_{a} p( j \in \mathcal{G}_k | a, g_j) da \biggr) 
\frac{ e^{ -\frac{ (a_i - \mu_{kg_j})^2 }{ 2(\lambda_{g_ig_j} + \sigma_{kg_j}^2) } } }{ \sqrt{ 2\pi(\lambda_{g_ig_j} + \sigma_{kg_j}^2) } },
\end{align}
where we take the probability of alter $j$ being in group $\scriptG_k$ given alter's age $a_j$ and gender $g_j$, normalize it with respect to the alter's age, approximate the normalized quantity with its discrete age counterpart, and approximate the discrete age normalized quantity with a Normal distribution so that:
\begin{align}
\frac{ p( j \in \mathcal{G}_k | a_j, g_j) }{ \int_{a} p( j \in \mathcal{G}_k | a, g_j) da } 
&\approx  \frac{ N_{k, a_j, g_j} / N_{a_j, g_j} } { \sum_a N_{k, a, g_j} / N_{a, g_j} } && \label{eq:subpopulation_normal_approx} \\\nonumber
&\approx  \normal(a_j | \mu_{kg_j}, \sigma_{kg_j}^2).
\end{align}
The approximating normal's center $\mu_{kg_j}$ and scale $\sigma_{kg_j}^2$ are estimated from $\mathcal{G}_k$'s population distributions $\frac{N_{k,a_j,g_j}}{N_{a_j,g_j}}$. These quantities are analogous, though not exactly equal, to the center and scale of the age distribution of individuals in $\mathcal{G}_k$ with gender $g_j$. 

In practice, the survey respondents' ages are usually observed discretely. In such cases, we can approximate the integral $\int_{a} p( j \in \mathcal{G}_k | a, g_j) da$ by the summation $\sum_\alpha \prob( j \in \mathcal{G}_k | \alpha, g_j)$ over discrete age $\alpha \in \{0,1,2,..\}$ so that the expression becomes
\begin{align}
\mu_{ik} 
&= d_i \sum_{g_j} \rho_{g_ig_j} 
\biggl( \int_{a} p( j \in \mathcal{G}_k | a, g_j) da \biggr) 
\frac{ e^{ -\frac{ (a_i - \mu_{kg_j})^2 }{ 2(\lambda_{g_ig_j} + \sigma_{kg_j}^2) } } }{ \sqrt{ 2\pi(\lambda_{g_ig_j} + \sigma_{kg_j}^2) } } && \\\nonumber
&\approx d_i \sum_{g_j} \rho_{g_ig_j} 
\biggl( \sum_\alpha \prob( j \in \mathcal{G}_k | \alpha, g_j) \biggr) 
\frac{ e^{ -\frac{ (a_i - \mu_{kg_j})^2 }{ 2(\lambda_{g_ig_j} + \sigma_{kg_j}^2) } } }{ \sqrt{ 2\pi(\lambda_{g_ig_j} + \sigma_{kg_j}^2) } } && \\\nonumber
&= d_i \sum_{g_j} \rho_{g_ig_j} 
\biggl( \sum_{a} \frac{ N_{k, a, g_j} }{ N_{a, g_j} } \biggr) 
\frac{ e^{ -\frac{ (a_i - \mu_{kg_j})^2 }{ 2(\lambda_{g_ig_j} + \sigma_{kg_j}^2) } } }{ \sqrt{ 2\pi(\lambda_{g_ig_j} + \sigma_{kg_j}^2) } },
\end{align}

Compared to the unrestricted elements of the \citet{McCormick+others:2010} mixing matrix in Equation \ref{eq:nonrandom_mixing_matrix}, the latent kernel framework allows us to have more control over the shape of the distributions in Figure \ref{fig:nonrandom_mixing_matrix}. Additionally, this framework allows us to estimate the alter demographic distributions for egos of any arbitrary age, rather than just for egos in 6 age categories.

This latent mixing kernel framework is also more parsimonious. Indeed, our gender matrix $\rho_{2\times2}$ only requires estimation of $2$ parameters and the mixing kernel only requires estimation of $4$ kernel bandwidths $\lambda_{g_ig_j}$. Consequently, this framework requires estimation of only $6$ parameters whereas the \citet{McCormick+others:2010} mixing matrix requires estimation of $42$ free parameters.

% Dependence on Alter Age
\subsection{Dependence on Alter Degree}
\label{subsec:alter_age_dependence}

We also considered modeling the dependence of mixing on alter degree $d_j$, but found that the dependence does not exist because $d_j$ is integrated out of the expression. The proof is as follows:
\begin{align}
\mu_{ik} 
&= d_i \sum_{g_j} \int_{a_j} \int_{d_j} p( j \in \mathcal{G}_k, a_j, g_j, d_j | a_i, g_i, i \to j) dd_j da_j && \\\nonumber
&= d_i \sum_{g_j} \int_{a_j} \int_{d_j} p( j \in \mathcal{G}_k | a_j, g_j, d_j, a_i, g_i, i \to j) p(a_j, g_j, d_j | a_i, g_i, i \to j)  dd_j da_j && \\\nonumber
&= d_i \sum_{g_j} \int_{a_j} \int_{d_j} p( j \in \mathcal{G}_k | a_j, g_j) p(a_j, g_j | a_i, g_i, i \to j) p(d_j | a_j, g_j, a_i, g_i, i \to j) 
 dd_j da_j && \\\nonumber
&= d_i \sum_{g_j} \int_{a_j}  p( j \in \mathcal{G}_k | a_j, g_j) p(a_j, g_j | a_i, g_i, i \to j) 
\biggl( \int_{d_j}  p(d_j | a_j, g_j, a_i, g_i, i \to j) dd_j \biggr) da_j && \\\nonumber
&= d_i \sum_{g_j} \int_{a_j}  p( j \in \mathcal{G}_k | a_j, g_j)  p(a_j, g_j | a_i, g_i, i \to j)  da_j && \\\nonumber
&= \dots && \\\nonumber
&= d_i \sum_{g_j} \rho_{g_ig_j} 
\biggl( \sum_{a} \frac{ N_{k, a, g_j} }{ N_{a, g_j} } \biggr) 
\frac{ e^{ -\frac{ (a_i - \mu_{kg_j})^2 }{ 2(\lambda_{g_ig_j} + \sigma_{kg_j}^2) } } }{ \sqrt{ 2\pi(\lambda_{g_ig_j} + \sigma_{kg_j}^2) } }.
\end{align}

% Adjustment for Varying Kernel Bandwidth
\subsection{Kernel Bandwidth Spline}
\label{subsec:kernel_bandwidth_spline}

While the latent mixing kernel framework allows us to impose structure on the social mixing patterns between subpopulations, it is also restrictive because the scales $\lambda_{g_ig_j}$ of the alter age distributions depend only on ego and alter gender, and not on ego age. This is a strong assumption because it is unlikely that, for example, a female 20-year-old's male acquaintances are as tightly concentrated around age 20 as are a female 70-year-old's male acquaintances are around age 70. Indeed, it seems reasonable that the age spread of one's acquaintanceships may depend on his or her age.

To resolve this issue, we extend the model in Section \ref{subsec:nonrandom_mixing_kernel} by allowing the kernel bandwidth to be a function of not only discrete gender but also continuous ego age $\lambda_{g_ig_j}(a_i)$. Because this bandwidth still does not depend on alter age $a_j$, it is a constant with respect to the integral in Derivation \ref{eq:nonrandom_mixing_kernel_derivation}. As such, the solution of the integral remains unchanged with respect to all of the other terms in the derivation, and we can essentially replace $\lambda_{g_ig_j}$ with $\lambda_{g_ig_j}(a_i)$. The resulting model is
\begin{align}
\mu_{ik} 
&= d_i \sum_{g_j} \rho_{g_ig_j} 
\biggl( \sum_{a} \frac{ N_{k, a, g_j} }{ N_{a, g_j} } \biggr) 
\frac{ e^{ -\frac{ (a_i - \mu_{kg_j})^2 }{ 2(\lambda_{g_ig_j}(a_i) + \sigma_{kg_j}^2) } } }{ \sqrt{ 2\pi(\lambda_{g_ig_j}(a_i) + \sigma_{kg_j}^2) } },
&& \label{eq:nonrandom_mixing_kernel_with_spline} 
\end{align}
where we also impose structure on $\lambda_{g_ig_j}(a_i)$ itself so that the model remains tractable. 

In order to keep $\lambda_{g_ig_j}(a_i)$ flexible without drastically increasing parameter complexity, we model the bandwidth as a spline (\citet{DeBoor:1978}) in $a_j$. Splines are a linear combination of basis splines, or B-splines, that are uniquely defined by two parameters: (i) the polynomial degree, $p$; and (ii) a non-decreasing sequence of knots, $t_1, \dots , t_q$, defined over the input range of the data that is being fit. The order of a spline family is defined as $p + 1$. B-splines of order 1 $(p = 0)$ are a set of piece-wise constant functions
\begin{align}
B_{n,1}(x) := 
\begin{cases}
1, \text{ if } & t_n \leq x < t_{n+1} \\
0, & \text{otherwise, }
\end{cases}
\end{align}
where $B_{n,k}$ denotes the $n^{\text{th}}$ member of a family of B-splines of order $k$ (or equivalently of degree $k-1$). B-splines of higher orders are defined recursively as
\begin{align}
B_{n,k}(x) := w_{n,k} B_{n,k-1}(x) + (1 - w_{n+1,k})B_{n+1,k-1}(x),
\end{align}
where
\begin{align}
w_{n,k} := 
\begin{cases}
\frac{x-t_n}{t_{n+k-1}-t_n}, \text{ if } & t_n \neq t_{n+k-1} \\
0, & \text{otherwise.}
\end{cases}
\end{align}
Thus, at a given point, $x$, a B-spline function of order $k$ is a linear combination of two B-splines of order $k-1$. Consequently, a spline of order $k$ (degree $k-1$) with knot sequence ${\bf t}$ $= t_1,\dots,t_q$ is defined as a linear combination of the B-splines, $B_{n,k}$, corresponding with that knot sequence. The set of all such splines can be denoted as
\begin{align}
S_{k,t}(x) &= \biggl\{  \sum_{n=1}^N \alpha_n B_{n,k}(x), \alpha_n \in \mathbb{R} \biggr\} \\ \nonumber
N &= |{\bf t}| + k - 2,
\end{align}
where $N$, the number of basis functions, is equal to the number of knots $|{\bf t}|$ plus the order $k$ minus 2.

For our particular implementation, we use a fourth order spline with the knots set to the deciles of the population age distributions (i.e. $|{\bf t}| = 11$ so that $N = 11 + 4 - 2 = 13$). With this setup, and after adding an intercept term, the expression for the kernel bandwidth becomes
\begin{align}
\lambda_{g_ig_j}(a_i) &= \alpha_0^{g_ig_j} + \sum_{n=1}^{13} \alpha_n^{g_ig_j} B_{n,k}(a_i). && \label{eq:kernel_spline}
\end{align}

Using the spline framework then, there are $14$ coefficients to estimate for each of the four splines $\lambda_{g_ig_j}(a_i)$, resulting in $56$ parameters. The total number of parameters to estimate for this model is then $58$, compared to $6$ in the base model of Section \ref{subsec:nonrandom_mixing_kernel} and $42$ for the \citet{McCormick+others:2010} discrete mixing matrix. Because this model allows us to estimate the specific bandwidth value for any continuous ego age, however, we believe the added parameter complexity is worthwhile.

In general, we recommend modeling the bandwidth's dependence on categorical variables (e.g. gender, race) by estimating a separate bandwidth for each category; while the bandwidth's dependence on continuous covariates should be estimated using splines. 
