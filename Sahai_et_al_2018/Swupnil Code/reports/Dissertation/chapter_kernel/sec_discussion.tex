\section{Discussion}
\label{sec:kernel_discussion}

This paper presents a latent kernel framework for social mixing using ARD. We show that, through the choice of a particular family of models, we can produce an interpretable representation of latent mixing patterns in survey data. 

In scaling up the ARD to the full network we make assumptions about respondents' abilities to recall their network. First, we assume accurate recall from respondents' complete networks, which is typically not valid for moderate to large subpopulations (e.g. people named Michael). We also assume that the respondent has accurate information about the group membership of each of their alters. This issue, known in sociology literature as transmission errors, is more common with some groups than others (e.g. acquaintances of a woman who has had an abortion may not know the woman's status). In some cases it is possible to select subpopulations that minimize transmission errors (e.g. first names), yet this remains an open problem in cases where groups of interest are prone to transmission errors (e.g. stigmatized occupations).

Recent work demonstrates that features of social mixing, such as homophily (tendency for actors to form ties with similar others), are distinguishable after aggregation. In particular, \citet{McCormick+others:2010} estimate mixing patterns using an unstructured mixing matrix that requires age to be binned into categories. The latent kernel model, in contrast, provides a structured, yet flexible, framework to estimate social mixing between individuals of any age and does not suffer from identifiability issues. The addition of the kernel bandwidth spline provides further insights by estimating social mixing variability granularly over time. 

In using a continuous framework for the age of the agents in a social network, we also make assumptions about various age distributions. Firstly, we assume that the normalized (with respect to discrete age) frequency of the subpopulations amongst individuals of a particular age and gender can be approximated by a normal distribution. Secondly, we assume that the Gaussian kernel need not be truncated to account for the fact that individuals cannot have a negative or very large (100+) age. If these assumptions were relaxed, the degree and social mixing estimates would likely align more closely with reality. However, the computational costs of evaluating a non-closed form negative binomial expectation could be immense, particularly in an MCMC framework. 

The Gaussian kernel, however, still provides room for future flexibility despite its unimodality and symmetry. Mixtures of Gaussian distributions, for example, would provide a more flexible representation of latent features and could provide additional insights into inter-generational mixing patterns, while still maintaining computational tractability. The bandwidth spline framework could also be applied to the gender mixing rates to provide insight into how gender mixing changes by age. At the same time, additional categorial variables could be used in place of gender. For example, a spline framework applied to race could be used to understand social mixing rates between races over time.