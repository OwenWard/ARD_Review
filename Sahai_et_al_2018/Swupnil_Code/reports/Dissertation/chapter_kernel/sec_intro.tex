\section{Introduction}
\label{sec:kernel_introduction}

Currently, the dominant framework for empirical social science is the sample survey. Such surveys, however, have made it traditionally difficult to understand the composition of social networks. Indeed, sample surveys been have described by \citet{Barton:1968} as a "meatgrinder" that completely removes people from their social contexts. This is unfortunate because social networks have become an increasingly common framework for understanding and explaining social phenomena. In recent years, however, an abundance of sophisticated models have shown how partially observed data, via sample surveys, can be used to accurately estimate properties about an individual's social network.

\citet{McCarty+others:2001} showed that Aggregated Relational Data (ARD), which ask respondents how many connections they have with members of a certain subpopulation (e.g. How many individuals do you know who are gay?), can be used to estimate personal network size via the scale-up method. \citet{Zheng+others:2006} further expanded the limits of ARD by introducing overdispersion into the scale-up method as a way to estimate non-random social mixing. Most recently, \citet{McCormick+others:2010} introduced the latent non-random mixing matrix framework as a way to quantify the mixing patterns between people from different age groups and genders using ARD. It is this last model that we use as inspiration to develop a more powerful, yet statistically robust, framework for estimating social mixing patterns from partially observed data.

In particular, this paper extends the discrete \citet{McCormick+others:2010} mixing matrix into a continuous, structured framework by deriving a latent kernel representation of social mixing patterns. Instead of binning ego and alter ages into categories, which is more conducive to a discrete mixing matrix, we treat age as continuous. Furthermore, we replace the discrete rows of the mixing matrix with a continuous mixing kernel whose center is equal to the ego's age. Lastly, we allow the scale (bandwidth) of the mixing kernels to depend not only on ego and alter gender, but also on ego age. The result is a pair of gender-specific kernels, each uniquely defined for an ego's given age and gender, that encapsulate the age distributions of the ego's female and male acquaintanceships.

This paper is organized as follows. We begin in Section \ref{sec:kernel_previous_research} with a review of previous attempts to measure personal network size and social mixing patterns, focusing on the latent non-random mixing matrix of \citet{McCormick+others:2010} which is promising, but suffers from instability in estimating the rows of the discrete mixing matrix. In Section \ref{sec:kernel_derivation} we derive a continuous latent mixing kernel framework which resolves these problems, and as a byproduct enables estimation of the mixing kernel bandwidth as a function of age. In Section \ref{sec:kernel_model} we outline the priors and likelihood of our model, and describe our model fitting algorithm. In Section \ref{sec:kernel_results} we then discuss the results of fitting the model to 1,190 survey responses from an online survey we designed, asking respondents how many people they know with certain first names or occupations. In Section \ref{sec:kernel_discussion}, we draw on insights developed during the statistical modeling to offer guidelines for future improvements.