%----------------------------------------------------------------------------------------
% ABSTRACT
%----------------------------------------------------------------------------------------

This thesis includes three parts. The overarching theme is how to analyze structured hierarchical data, with applications to astronomy and sociology. The first part discusses how expectation propagation can be used to parallelize the computation when fitting big hierarchical bayesian models. This methodology is then used to fit a novel, nonlinear mixture model to ultraviolet radiation from various regions of the observable universe. The second part discusses how the Stan probabilistic programming language can be used to numerically integrate terms in a hierarchical bayesian model. This technique is demonstrated on supernovae data to significantly speed up convergence to the posterior distribution compared to a previous study that used a Gibbs-type sampler. The third part builds a formal latent kernel representation for aggregate relational data as a way to more robustly estimate the mixing characteristics of agents in a network. In particular, the framework is applied to sociology surveys to estimate, as a function of ego age, the age and sex composition of the personal networks of individuals in the United States.