Hierarchical Bayesian modeling of naturally partitioned data has become ubiquitous in social and physical sciences. For example, social surveys intrinsically have respondents belonging to different age groups, race groups, income levels, occupations, etc. Data collections in astronomy, on the other hand, typically involve distant stars and planets that belong to different subregions of the universe, such as galaxies. Hierarchical models, by partially pooling group specific coefficients (parametric cases) or functional forms (nonparametric cases), provide a natural way to regularize inferential results on such data sets, particularly when the number of hierarchical groups is large and the data models are sufficiently complex.

In this thesis, we discuss three projects related to partitioned datasets in astronomy and sociology, each constituting one chapter. In Chapter \ref{chap:ep}, which is based on \citep{Gelman+others:2017}, we discuss distributed expectation propagation (EP) as a framework for hierarchical Bayesian modeling of partitioned data. While distributed computing has become an active area of research in recent years, such methods have had difficulty appropriately handling the prior distribution for Bayesian inference. We demonstrate, however, that EP combined with MCMC for tilted distribution approximation can be used as a natural distributed Bayesian inference tool for partitioned data. We further discuss how distributed EP provides massive computational gains over the standard full MCMC approach, which aids in more efficiently modeling the relationship between infrared and ultraviolet radiation from distant stars. 

In Chapter \ref{chap:ode}, the motivating example is an existing data set of $740$ supernovae, previously analyzed in a hierarchical Bayesian model implemented with a Gibbs-type sampler in Python. We show how the Stan probabilistic programming language, combined with a No U-Turn Sampler and a transformation of the priors into an unconstrained space, allows us to estimate the expansion rate of the universe from the same data set in a much faster C++ implementation. Furthermore, we show how our probabilistic programming paradigm results in significantly less code being written for the sampler, while increasing the model's readability and mutability.

In Chapter \ref{chap:kernel} we introduce a novel latent kernel representation for aggregate relational data (ARD) as a way to more robustly estimate social mixing patterns. While previous work has allowed estimation of social mixing between discrete subgroups of a social network, our framework extends current approaches to allow estimation of social mixing between individuals of continuous characteristics, such as age. We further expand on our novel representation by allowing the kernel's bandwidth to be modeled by a continuous spline (rather than a fixed constant), which allows us to estimate, for the first time, how the age-based homophily of cross-gender social mixing changes by age. Lastly, we apply our methodology to an online survey and demonstrate how these new social mixing patterns can be estimated from ARD regarding the names and occupations of the respondents' social networks.  

Partitioned data provide both opportunities and challenges for statistical analysis of astronomical and social research. We hope that the three essays here contribute to our understanding and toolbox of tackling hierarchical Bayesian modeling of partitioned data.